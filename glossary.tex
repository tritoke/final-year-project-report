% glossary entries
\newglossaryentry{ocrypto} {
  name={Online Cryptography},
  description={
    Online cryptography is where interactions with the cryptosystem are only possible
    via real-time interactions with the server. Primarily this is to prevent offline computation%
  }
}

\newglossaryentry{symcrypto} {
  name={Symmetric Cryptography},
  description={
    Symmetric Cryptography is where the both the sender and receiver share the same secret key. It is normally computationally more efficient, the most common such scheme is \gls{aes}%
  }
}

\newglossaryentry{asymcrypto} {
  name={Asymmetric Cryptography},
  description={
    Asymmetric Cryptography is where the the sender and receiver each have two keys - a public key which can be freely shared, and a private key which must be kept secret. Common examples of this are the RSA scheme and the various DH flavours%
  }
}

\newglossaryentry{bpake} {
  name={Balanced PAKE},
  description={
    A Balanced PAKE is one in which both parties share knowledge the same secret.
    This is in contrast to other schemes such as Verifier-based/Augmented PAKEs.
  }
}

\newglossaryentry{apake} {
  name={Augmented PAKE},
  description={
    A Balanced PAKE is one in which both parties share knowledge the same secret.
    This is in contrast to other schemes such as Verifier-based/Augmented PAKEs.
  }
}

\newglossaryentry{sgprime} {
  name={Sophie Germain Prime},
  text={Sophie Germain prime},
  description={
    A number $n$ is a Sophie Germain Prime if $n$ is prime and the $2n+1$ is also prime.
  }
}

\newglossaryentry{safeprime} {
  name={Safe Prime},
  text={safe prime},
  description={
    A number $2n+1$ is a Safe Prime if $n$ is prime, it is the effectively the other part of a Sophie Germain prime.
  }
}

\newglossaryentry{veri} {
  name={Verifier},
  text={verifier},
  description={
    A representation of the user's password put through some one-way function.
    This could be as simple as just storing a hash of the password, though for most PAKEs the verifier is an element of whatever group we are working in.
    An example can be seen on page \pageref{text:srp-verifier-generation}%
  }
}

\newglossaryentry{abgroup} {
  name={Abelian Group},
  description={
    A group whose operator is also commutative. e.g. Addition over $\ZZ$.\
  }
}

% acronyms
\newacronym{aes}{AES}{Advanced Encryption Scheme}
\newacronym{aucpace}{AuCPace}{Augmented Composable Password Authenticated Connection Establishment}
\newacronym{dh}{DH}{Diffie-Hellman}
\newacronym{eap}{EAP}{Extensible Authentication Protocol}
\newacronym{ecc}{ECC}{Elliptic Curve Cryptography}
\newacronym{eke}{EKE}{Encrypted Key Exchange}
\newacronym{iiot}{IIOT}{Industrial Internet of Things}
\newacronym{iot}{IOT}{Internet of Things}
\newacronym{tls}{TLS}{Transport Layer Security}
\newacronym{mcu}{MCU}{Microcontroller}
\newacronym{pake}{PAKE}{Password-Authenticated Key-Exchange}
\newacronym{rsa}{RSA}{Rivest-Shamir-Adleman}
\newacronym{srp}{SRP}{Secure Remote Password}
\newacronym{nist}{NIST}{National Institute of Standards and Technology}
\newacronym{vpake}{V-PAKE}{Verifier based PAKE}
\newacronym{spake}{SPAKE}{Simple Password-Authenticated Key-Exchange}
