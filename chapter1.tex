\chapter{Introduction}
\label{chap:intro}

Today's standards for Encryption and Authentication are fundamentally broken.
The age old standard practice of obtaining an encrypted connection to a server, then sending a cleartext password to authenticate users is flawed.
\glspl{pake} are a profoundly different way of looking at how encryption and authentication are performed, and could represent the basis for a safer and more secure future even in the face of the ever increasing threat quantum computers represent to encryption.

\section{Aims}
This project aims to understand the nature of how \glspl{pake} work, what they can and cannot do, and how they can be used to improve the security of almost all password based authentication systems.
The "quantum annoying" property of \glspl{pake} will also be explored, to understand how \glspl{pake} can be used to delay the need for post-quantum cryptography by years or even decades \cite{quantum-annoying}.
Additionally an implementation of a modern \gls{pake} protocol (\gls{aucpace}) will be created and and contributed back to the RustCrypto open source project.

\section{Deliverables}
The implementation of the \gls{aucpace} protocol in Rust is the core deliverable for this project.
Performance of the library will be compared against those for other \gls{pake} protocols.
The metrics of execution time and code size will be used.

\section{Challenges}
Working with elliptic curve cryptography proved to be particularly challenging.
A lack of understanding of one particular area proved fatal when a catastrophic bug was found in the codebase late in the project.
\Cref{chap:testing} talks about remediating this bug and ensuring it cannot happen again.

\section{Structure}
The report will follow the following structure:
\begin{itemize}
    \item{\Cref{chap:intro} introduces the project and provides an explanation of the objectives.}
    \item{\Cref{chap:context} goes into detail on the history of \gls{pake} algorithms and elliptic curve cryptography.}
    \item{\Cref{chap:design} explains the design decisions made around the implementation of \gls{aucpace}.}
    \item{\Cref{chap:impl}}
    \item{\Cref{chap:testing}}
    \item{\Cref{chap:conclusion} wraps up the project, reflecting on the project and giving a conclusion.}
\end{itemize}
